\documentclass{report}
\usepackage[utf8]{inputenc}
\usepackage[francais]{babel}
\usepackage[T1]{fontenc}
\usepackage{lmodern}
\usepackage{ifpdf}
\usepackage{graphicx}
\usepackage{geometry}
\renewcommand{\familydefault}{\sfdefault}

\geometry{hmargin=50pt, vmargin=50pt}

\title{Rapport de projet : Tableau virtuel interactif}
\author{Baptiste Saleil \and Geoffrey Mélia \and Julien Pagès \and Kevin Bollini}
%faudrai mettre une page/paraph qq part, pour marquer nos formations, que ca vienne pas comme un cheveux ds la soupe au milieu du rapport
\date{\today}
\ifpdf
	\pdfinfo 
	{
		/Author (bsaleil,gmelia,jpages,kbollini)
		/Title (Rapport de projet)
		/Subject (Tableau virtuel interactif)
		/Keywords ()
		/CreationDate (\today)
	}
\fi

\begin{document}
	% Page de titre
	\maketitle
	\thispagestyle{empty}
	\newpage
	
	% Sommaire
	\tableofcontents

	%Table des figures
	\listoffigures
	
	\newpage 
	Remerciements : \\
	
	Nous tenons à remercier tout particulièrement M. William Puech (enseignant chercheur au LIRMM à l'université Montpellier 2, responsable de la formation du Master informatique IMAGINA) sans qui ce projet n'aurait pas pu se faire, pour avoir accepter de nous encadrer, pour son aide et son implication.\\
	Nous souhaitons aussi remercier M. Benoit Lange (doctorant au LIRMM) pour ses conseils et sa présence au cours de ce projet. \\
	\newpage
	\chapter{Introduction}
		\section{Présentation}
		Les cinq dernières années ont marqué un grand renouveau dans les interfaces entre hommes et terminaux (écrans tactiles, contrôles vocaux, détection de mouvements). Cet état de faits nous a amenés à penser une application mettant en scène l'une de ces nouvelles façons d'interagir.\\
		 L'objectif du projet issu de cette réflexion est donc de créer un tableau virtuel, avec lequel une ou plusieurs personnes peuvent interagir en réalisant directement des gestes comme sur une toile réelle, par reconnaissance des mouvements detectés par une webcam. \\

		\begin{figure}[!h]
			\centering
			\includegraphics[scale=0.3]{../soutenance/capture-intro.png}\\
			\caption{Aperçu de l'application}
			\label{Aperçu de l'application}
		\end{figure}
	\newpage
		\section{Contexte}
		% TODO
		Ce projet n'est pas un simple choix dans une liste de projets, en effet, il est à l'initiative de notre groupe.\\
		Nous avons décidé de proposer ce sujet pour diverses raisons, dont voici les pricipales:
		\begin{itemize}
			\item Tout d'abord, c'est un projet que nous avions en tête depuis plus d'un an, qui a donc eu le temps de murîr et d'être au niveau d'un TER.
			\item Deuxièmement, notre groupe vient de deux formations distinctes : AIGLE, formation orientée génie Logiciel ; et IMAGINA, spécialisée dans le jeu vidéo et l'image. Ce projet représentait une manière opportune de lier ces deux spécialités
			\item Enfin, ce projet était pour nous une bonne occasion de réaliser une application fonctionnelle et distribuable, et par conséquent nous permettre d'explorer une autre facette du développement d'applications : la mise en production.

		\end{itemize}	
	\chapter{Analyse et conception}
		\section{Étude de l'existant et faisabilité}
		\subsubsection{Faisabilité}
		% TODO : expliquer si ça nous parait possible 
		Notre groupe ayant lui-même proposé ce sujet, celui-ci s'accorde parfaitement à nos formations et spécialités. Le choix de ce sujet a donc été réalisé en fonction de nos expériences, savoir-faire et affinités.\\
Ce projet s'inscrivant dans le cadre d'un TER de notre formation, l'étude de la faisabilité n'incluera pas certains critères comme l'étude de marché, le contexte économique ou le besoin réel. 
Nous nous sommes concentrés notamment sur les compétences techniques et de gestion. Pour les parties financement et analyse des coûts, il va sans dire que nous ne disposions d'aucun financement et n'avons donc utilisé que des outils gratuits. \\

		\subsubsection{Existant}
		La vision par ordinateur et particulièrement le suivi d'objets, sont des domaines connus et pour lesquels il existe de nombreux travaux. Nous pouvons nous inspirer de certains de ces travaux afin de proposer des fonctionnalités plus pertinentes, éviter certains écueuils ou plus directement utiliser des outils existants comme la librairie OpenCV.\\
			\begin{itemize}
				% TODO : détailler un peu plus l'existant
				\item Exemples dans le jeux vidéo : Kinect, Eye-toy, CamSpace (techniques pour l'IHM, par exemple, mouvements continus ou immobilité sur une zone).
				\item Exemples issus de thèses et de projets de recherche (techniques de programmations).
			\end{itemize}
			\begin{figure}[!h]
				\centering
				\includegraphics[scale=0.2]{kinect_screen_lg.jpg}\\
				\caption{Exemple du Kinect}
				\label{Exemple du Kinect}
			\end{figure}
		\newpage
		\section{Gestion du projet}
			\subsection{Choix stratégiques}
			Lors de ce projet, nous avons distingué deux parties majeures.
			La première est une bibliothèque de suivi d'objets. La seconde, une application qui utilise cette bibliothèque pour réaliser un logiciel de dessin interactif. \\
			Ce choix présente plusieurs intérêts : \\
			\begin{itemize}
				\item{Premièrement, ce découpage permet de bien différencier les tâches. Le développement de la librairie requiert des connaissances en traitement d'images pour tracer des couleurs/objets, appliquer des filtres, manipuler les images, etc. Ceci correspond parfaitement à la formation de deux membres du groupe(formation IMAGINA à l'UM2). 
				Le développement de l'application requiert quant à lui des connaissances en génie logiciel, pour l'architecture de l'application, en IHM, pour la réalisation de l'interface (logicielle et gestuelle), ou encore en réseau pour la mise en place de l'architecture client/serveur. 
				Ces différents points correspondent également parfaitement à la formation des deux autres membres du groupe. (Formation AIGLE à l'UM2). Ainsi, les tâches peuvent bien se répartir en fonction des connaissances et spécialités de chacun.} \\
				\item{Deuxièmement, l'intérêt de ce découpage est d'avoir deux projets à la fois complètement indépendants et complémentaires. En effet, la librairie est développée offre des fonctionnalités de suivi utilisables par n'importe quelle application.\\
				L'application se charger d'utiliser les fonctionnalités de traitement d'images de la librairie en ajoutant une interface graphique, une couche réseau, une connexion aux webcams, la récupération des images etc.
				La librairie et l'application sont donc deux projets développés en parallèle et très modulaires. Notre application peut très bien utiliser une autre librairie de traitement d'images, et la librairie peut très bien être utilisée par d'autres applications qui offriraient des fonctionnalités totalement différentes de la nôtre.}
			\end{itemize}
			\newpage
			\subsection{Diagramme de Gantt}
			Étant donné que nous avons proposé notre propre sujet, la première étape a été de rédiger un cahier des charges pour définir ce que nous voulions faire et de quelle manière. \\
			Après ceci, nous avons commencé à effectuer des recherches du côté bibliothèque et application, pour savoir quels outils et techniques utiliser pour arriver à nos objectifs. 
		  Nous avons réservé plusieurs semaines pour effectuer des recherches et perfectionner notre idée du projet. Nous avons donc lu des articles de recherches sur des thématiques proches, effectuer des essais d'outils etc. Cette période d'analyse et de conception sert aussi à penser l'application en imaginant des cas d'utilisations (use cases) et en réalisant le diagramme de classes. \\
			
			Par la suite, la majeure partie du temps est réservée au développement en lui même, en prenant en compte la date de présentation du projet pour avoir une réalisation complète. \\  
			Bien entendu ces temps de développement s'accompagnent de réunions avec tout le groupe de projet ainsi que de travail pour lier nos deux sous-parties et les tester ensemble. \\
			Des documents comme le rapport ou la documentation furent produit au fur et à mesure du projet, pour rester en adéquation avec le travail accompli. Voici le rétroplanning qui illustre le déroulement chronologique du projet : \\
				\begin{figure}[!h]
					\centering
					\includegraphics[scale=0.6]{../soutenance/retro-planning.pdf}\\
					\caption{Rétroplanning}
					\label{Rétroplanning}
				\end{figure}
		\newpage
		\section{Outils utilisés}
		Durant la période de recherche et au cours du projet nous avons utilisé différent outils (au sens large) pour nous aider dans la réalisation. Nous les détaillons ici dans cette section. \\
		
			\subsubsection{Outils collaboratifs}
			Tout d'abord nous avons utilisés des outils pour nous aider à synchroniser notre travail entre les deux sous-groupes. \\ \ \\
			\includegraphics[scale=1]{../logos/subversion-logo.png} 
			Subversion : Un gestionnaire de versions qui nous permettait de synchroniser notre travail, de partager le code source et de s'échanger quelques documents complexes (ce rapport par exemple).  \\
		
			\includegraphics[scale=0.25]{../logos/Gobby-logo.png} 
			Gobby :
			Logiciel libre pour l'édition collaborative, qui nous permets de développer à distance sur le même document simultanément. \\
			Bien sûr ces outils ne sont pas suffisants, nous avons donc fait des réunions régulières pour organiser le travail et décider des orientations, par ailleurs, nous avons fais beaucoup d'échanges pour que tout le monde soit toujours au courant des différentes communications importantes. \\
			
			\subsubsection{Outils techniques}
			
			Langage C pour la bibliothèque : ce langage est bien adapté pour écrire une bibliothèque car il nous permets de faire des fonctions, ce qui était dans notre idée ce qu'il fallait pour cette bibliothèque. Par ailleurs c'est un langage bien adapté pour faire du traitement de l'image car il est de bas niveau, et très rapide. De plus nous avons utilisé par la suite des librairies (OpenCV par exemple) qui ont été écrites initialement en C. \\
			Langage C++ pour l'application : c'est un langage objet ce qui nous aidait pour la conception, il est également très rapide et c'est un langage avec lequel nous sommes à l'aise et nous avons de l'expérience, ça nous évitait donc de dévoir réapprendre et donc de perdre du temps précieux.\\
			
			Pour ce projet nous utilisons plusieurs librairies pour nous aider : \\
			
			\includegraphics[scale=0.15]{../logos/qt_logo.png}
			Qt : librairie d'IHM (et bien plus) en C++, nous aide pour faire les interfaces graphiques, pour la partie réseau également où elle offre des abstractions agréables. \\
			
			\includegraphics[scale=1]{../logos/OpenCV_Logo.png}
			OpenCV : librairie de traitement de l'image, nous aide à exploiter la webcam et pour le traitement de l'image avec plusieurs fonctionnalités. \\
		
		\newpage
		\section{Analyse}
			\subsection{Cas d'utilisations}
				\begin{figure}[!h]
						\centering
						\includegraphics[scale=0.6]{../uml/Dessin.png}\\
						\caption{Cas d'utilisation dessin}
						\label{Cas d'utilisation}
				\end{figure}
			\newpage
			\subsection{Diagrammes de séquences}
			Les diagrammes qui suivent expliquent le fonctionnement de l'application pour une utilisation locale, et une utilisation en réseau.
				\begin{figure}[!h]
						\centering
						\includegraphics[scale=0.6]{../soutenance/sequence_local.png}\\
						\caption{Diagramme de séquence local}
						\label{Diagramme de séquence local}
				\end{figure}
				On peut voir sur ce diagramme que le fonctionnement en local est assez basique. Dans un premier temps, l'utilisateur \textbf{étalonne} l'objet avec lequel il veut dessiner. Ensuite, lorsque l'utilisateur déplace l'objet en question, l'application \textbf{détecte} ce déplacement, et \textbf{dessine} à la nouvelle position de l'objet. Ce schéma de détection-dessin est répété jusqu'à fermeture de l'application. \\
				Ainsi, l'utilisateur verra se dessiner chacun des mouvements qu'il effectue avec son objet.

				\newpage
				\begin{figure}[!h]
						\centering
						\includegraphics[scale=0.6]{../soutenance/sequence_reseau.png}\\
						\caption{Diagramme de séquence réseau}
						\label{Diagramme de séquence réseau}
				\end{figure}
				En réseau, l'utilisation suit le même schéma, mais un autre acteur entre en jeu, le \textbf{serveur}. \\
				Comme pour une utilisation locale, l'utilisateur comment par \textbf{étalonner} l'objet à suivre. Puis, la connexion s'effectue auprès du serveur, qui va renvoyer le dessin courant au nouveau client. Une fois en possession du dessin, le client le stocke et peut rentrer dans la boucle de détection-dessin : \\
				Comme en local, l'application va \textbf{détecter} le mouvement de l'objet suivi. La nouvelle position est envoyée au serveur qui va dessiner sur son propre tableau, puis renvoyer à tous les clients l'ordre de \textbf{dessiner} à cette même position. Chaque client verra donc ce que les autres clients dessinent. \\
				Ainsi, l'utilisateur verra se dessiner chacun des mouvements qu'il effectue avec son objet, mais égelement ceux de tous les autres clients.
				\\ $$$$ \\
				On peut donc voir que pour l'utilisateur, le fonctionnement est identique qu'il dessine seul, ou avec d'autres personnes :
				\begin{itemize}
					\item Il étalonne l'objet
					\item Il déplace l'objet
					\item l'objet dessine sur le tableau
				\end{itemize}
				
				Cependant, derrière, le fonctionnement n'est pas le même. En réseau, chaque mouvement passe par le serveur qui se charge de faire le lien avec tous les clients de l'application.

			\newpage
			\subsection{Diagramme de classes}
				La bibliothèque fonctionne grâce à une structure de données utilisée par deux fonctions enveloppes. \\
				Le curseur est représenté en mémoire sous la forme d'une strucure de données "Cursor". \\
				Cette structure de données est engendrée par la fonction "calibration" de la librairie. Elle est ensuite utilisée par la fonction "track", qui permet la mise à jour des informations relatives au curseur par rapport à une image. \\
				\begin{figure}[!h]
						\centering
						\includegraphics[scale=0.8]{../soutenance/libtrack-uml.png}\\
						\caption{Architrecture de la bibliothèque}
						\label{Architrecture de la bibliothèque}
				\end{figure}

				La bibliothèque dépend de deux bibliothèques externes:
				\begin{itemize}
					\item OpenCV, Spécialisée dans le traitement des images
					\item CvBlob, librairie issue d'OpenCV utlisée pour la recherche d'objet dans des images binaires
				\end{itemize}

				\newpage			

				\begin{figure}[!h]
						\centering
						\includegraphics[scale=0.6]{../uml/classes.png}\\
						\caption{Diagramme de classe de l'application}
						\label{Diagramme de classe de l'application}
				\end{figure}

				
	
	\chapter{Réalisation}
		\section{Bibliothèque de suivi}
			Cette partie du développement a pour objectif de fournir un ensemble de fonctionnalités permettant la reconnaissance et le suivi d'un objet, que nous appelerons ici Curseur, à partir d'images. \\
			L'objectif est de développer notre propre bibliothèque pour d'une part avoir un outil remplissant précisément nos objectifs et d'autre part de faire un travail de recherche et développement dans le domaine du traitement de l'image (en particulier le suivi d'objets). \\
			Un certain nombre de librairies existantes proposent des moyens nous facilitant la tâche, notamment grâce à des outils de reconnaissance de formes, de suivis de modèle ou d'évaluation de composantes connexes. 
			Le travail consiste donc à tirer partie de ces bibliothèques en les associant de manière à proposer des méthodes efficaces et simples à utiliser.
			Cela implique une bonne connaissance des outils existants, de leur compréhension et de leur utilisation conjointes dans le but de faire notre propre bibliothèque. \\
			La Bibliothèque de suivi sert à réaliser le suivi d'un objet, que nous appelerons ici Curseur, avec un minimum de connaissance en traitement des images.\\
			
			La bibliothèque fonctionne grâce à une structure de données manipulée par deux fonctions enveloppes. \\
			Le curseur est représenté en mémoire sous la forme d'une strucure de données "Cursor". \\
			Cette structure de données est engendrée par la fonction "calibration" de la librairie. Elle est ensuite utilisée par la fonction "track", qui permet la mise à jour des informations relativess au curseur.
					
				\newpage
				\subsubsection{Structure de données}
				\subsubsection{Fonctions}
			\subsection{}
		\section{Application}
	\paragraph{Application :\\}
L'application doit exploiter le plus efficacement possible la librairie de reconnaissance. Nous voulons donc mettre en place un étalonnage de l'objet à suivre le plus précisément possible. Le projet doit ensuite être exploitable et utile, nous devons donc mettre en place un certain nombre d'outils tels que la possibilité de gommer, changer de forme, de couleur etc. 
Un module réseau est aussi prévu pour une utilisation concurrente et synchronisée. Et bien sûr la fonctionnalité clé de l'application est le tableau virtuel, qui devra donc être exploitable et le dessin devra être fidèle aux mouvements reconnus.
	\chapter{Résultats}
		\section{Méthodes de suivi}
		\section{Application}
	
	\chapter{Conclusion}
		\section{Difficultés rencontrées}
		\section{Perspectives}
		\section{Conclusion}
	
	\chapter{Références}
	
	\part{Annexes}
	\appendix
		\chapter{Documentation de la librairie}	
\end{document}
