\documentclass{report}
\usepackage[utf8]{inputenc}
\usepackage[francais]{babel}
\usepackage[T1]{fontenc}
\usepackage{lmodern}
\usepackage{ifpdf}
\usepackage{graphicx}
\usepackage{geometry}
\renewcommand{\familydefault}{\sfdefault}

\geometry{hmargin=50pt, vmargin=50pt}

\title{Rapport de projet : Tableau virtuel interactif}
\author{Baptiste Saleil \and Geoffrey Mélia \and Julien Pagès \and Kevin Bollini}
\date{\today}
\ifpdf
	\pdfinfo 
	{
		/Author (bsaleil,gmelia,jpages,kbollini)
		/Title (Rapport de projet)
		/Subject (Tableau virtuel interactif)
		/Keywords ()
		/CreationDate (\today)
	}
\fi

\begin{document}
	% Page de titre
	\maketitle
	\thispagestyle{empty}
	\newpage
	
	% Sommaire
	\tableofcontents
	
	\newpage
	\chapter{Introduction}
		\section{Présentation}
		Le but principal du projet est de simuler une écriture ou un dessin sur un tableau virtuel interactif. \\
		Pour cela nous utiliserons une interface basée sur la reconnaissance de mouvements.
		
		\section{Contexte}
	
	\chapter{Analyse et conception}
		\section{Étude de l'existant et faisabilité}
		\section{Gestion du projet}
			\subsection{Choix stratégiques}
			\subsection{Diagramme de Gantt}
		\section{Outils utilisés}
		\section{Analyse}
			\subsection{Cas d'utilisations}
			\subsection{Diagramme de classes}
	
	\chapter{Réalisation}
		\section{Librairie}
		\section{Application}
	
	\chapter{Résultats}
		\section{Explications des algorithmes de suivi}
		\section{Application}
	
	\chapter{Conclusion}
		\section{Difficultés rencontrées}
		\section{Perspectives}
		\section{Conclusion}
	
	\chapter{Références}
	
	\part{Annexes}
	\appendix
		\chapter{Documentation de la librairie}	
\end{document}
