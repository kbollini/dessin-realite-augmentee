\documentclass{article}
\usepackage[utf8]{inputenc}
\usepackage[francais]{babel}
\usepackage[T1]{fontenc}
\usepackage{lmodern}
\usepackage{geometry}

\geometry{hmargin=50pt, vmargin=50pt}
\title{Bilan mi-parcours : IHM Ecriture sur Tableau Virtuel}
\author{Bollini Kevin, Mélia Geoffrey, Pagès Julien, Saleil Baptiste}

\date{\today}

\begin{document}
	\maketitle
	Nom du groupe : \textbf{PouerMouer}
	
	\section{Composition du groupe et contact}
	
	\begin{itemize}
	\item Kevin Bollini : kevin.bollini@etud.univ-montp2.fr \\
	\item Geoffrey Mélia : geoffrey.melia@etud.univ-montp2.fr \\
	\item Julien Pagès : julien.pages01@etud.univ-montp2.fr \\
	\item Baptiste Saleil : baptiste.saleil@etud.univ-montp2.fr \\
	\end{itemize}
	
	Pour rappel, nous travaillons en deux sous-groupe basés sur nos compétences, Kevin Bollini et Geoffrey Mélia travaillent
	sur la partie biliothèque de suivi d'objet car ils sont tous les deux en IMAGINA. \\
	La partie IHM et l'application qui utilise cette librairie est réalisée par Julien Pagès et Baptiste Saleil, tous les deux en AIGLE. \\
	\section{Tâches effectuées avec succès}
		\subsection{Partie bibliothèque}
		\begin{itemize}
		\item Développement d'une méthode de suivi avec la librairie cvBlob.
		\item Suivi par forme fonctionnel.
		\item Amélioration de l'étalonnage (basée sur une moyenne de couleur).
		\item Amélioration du suivi par couleur.
		\item Détection d'action lorsque on rapproche un objet.
		\item Optimisation de code et documentation.
		\end{itemize}
		
		\subsection{Application d'écriture virtuelle}
		\begin{itemize}
		\item Nouvelle interface d'étalonnage, avec possibilité de choisir la caméra, de faire un carré autour de l'objet, 
		puis de régler le seuil, et enfin choisir le mode réseau ou local.
		\item Mise en place du module réseau de l'application, plusieurs utilisateurs peuvent maintenant l'utiliser en même temps.
		\item Développement d'outils pour l'application tels que une gomme, le changement de couleur, choisir la taille
		du pinceau, exporter le dessin et mettre l'application en mode plein écran.
		\item Interface naturelle (reconnaissance de mouvements) pour changer d'outils, couleur etc.
		\item Grandes optimisation du code. 
		\item Refonte de l'interface de l'application.
		\end{itemize}
		
	\section{Problème rencontrés}
		\subsection{Partie librairie}
		\begin{itemize}
		\item Problèmes avec la gestion mémoire de la librairie OpenCV, qui n'est pas très claire et qui est implicite.
		\end{itemize}
		
		\subsection{Partie application}
	
\end{document}

